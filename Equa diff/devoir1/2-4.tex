Lagrange $(E): y = x \varphi(y') +\psi(y)$ \\ \\
1er cas: Si  $\varphi(y') = y'$ c-a-d $y = xy'+ \psi(y')$ \\
alors on pose $ p = y'$ \\
donc on a $ y' = xp + \varphi(p) \text{ et } y' = p + xp' + p'\psi'(p)$ \\
mais comme $ y' = p \text{ alors on a }  p = p + xp' + p'\psi(p)$ \\
ou encore $ 0 = p'(x + \psi'(p)) \text{ c-a-d } p' = 0 \text{ ou } x + \psi'(p) = 0$ \\
donc p: constante ou $ x = - \psi'(p)$ \\
Si p = y' est une constante alors $ y = Cx + K, K \in \mathbb{Q} \text{ ou si } x = -\psi'(p)$
alors $ y = -p\psi'(p) + \psi(p)$ ( car $y = xp + \varphi(p)$ )

\begin{align*}
	(E): y &= xy' + (y')^{3} \\
	\text{On pose } p = y' \\
	y = xp + p^3 &\implies y' = x + xp' + 3p^2p' \\
	y' = p &\implies p = x + p'x + 3p'p^2
\end{align*}

$\text{Or } p'(x + \psi'(p)) = 0 \quad p' = 0 \text{ ou } x-3p^2 = 0$ \\
\begin{align*}
	x &= -3p^2 \quad /y = xp + p^3 \\
	3p^2 &= -x \\
	p^2 &= -\frac{x}{3}
\end{align*}

\begin{align*}
	y &= xp + p(-\frac{x}{3}) \\
	  &= xp - \frac{x}{3}p \\
	y &= \frac{2xp}{3} \\
	p &= \frac{3y}{2x}
\end{align*}

\begin{align*}
	\text{Nous rappelons que } p^2 = -\frac{x}{3} \\
	(\frac{3y}{2x})^2 &= -\frac{x}{3} \\
	\frac{9y^2}{4x^2} &= -\frac{x}{3} \\
	27y^2 &= -4x^3 \\
	y^2 &= -\frac{4}{27}x^3
\end{align*}

$\text{Solution generale: }$ \\
$y = Cx + C^3, \quad C \in \mathbb{R}$ \\ \\
\boxed{y^2 = -\frac{4}{27} x^3}