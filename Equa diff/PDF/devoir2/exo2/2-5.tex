\underline{Forme de Lagrange} \\ \\
Une equation differentielle du premier ordre de Lagrange est donnee par :
\begin{equation}
    y = x \varphi(y') + \psi(y').
\end{equation}
Avec $p = y'$, on a :
\begin{align*}
    p &= x p + \sqrt{1 + p^2} \\
    p - x p &= \sqrt{1 + p^2} \\
    p (1 - x) &= \sqrt{1 + p^2} \\
    p^2 (1 - x)^2 &= 1 + p^2, \\
    p^2 - 2x p^2 + x^2 p^2 &= 1 + p^2, \\
    p^2 (x^2 - 2x) &= 1, \\
    p^2 &= \frac{1}{x^2 - 2x}, \\
\end{align*}

Avec
\[
\begin{array}{c|c|c|c|c}
    x & -\infty & 0 & 2 & +\infty \\
    \hline
    x & - & 0 & + & + \\
    \hline
    x - 2 & - & - & 0 & + \\
    \hline
    x(x-2) & + & - & - & + \\
\end{array}
\]
Pour que $p^2 > 0$ alors $x^2-2x = x(x-2) > 0$

On a
\begin{align*}
    x^2 -2x &= \frac{1}{p^2} \\
    x^2 -2x - \frac{1}{p^2} &= 0
\end{align*}

\begin{align*}
    \Delta &= (-2)^2 - 4 (\frac{1}{p^2}) \\
    &= 4 + \frac{4}{p^2} \\
    &= \frac{4p^2+4}{p^2} \\ 
    \Delta &= 4\frac{p^2 +1}{p^2} \\
    \sqrt\Delta &= \frac{2}{p}\sqrt{p^2+1}
\end{align*}

\begin{align*}
    x &= \frac{2 \pm \frac{2}{p}\sqrt{p^2+1} }{2} \\
    x &= \frac{2p \pm 2\sqrt{p^2+1}}{2p} \\
    x &= \frac{2p}{2p} \pm \frac{2\sqrt{p^2+1}}{2p} \\
    x &= 1 \pm 2\sqrt{ \frac{p^2}{p^2} + \frac{1}{p^2} } \\
    x &= 1 \pm 2\sqrt{ 1 + \frac{1}{p^2} } \\
\end{align*}
Alors  $x = 1 - 2\sqrt{ 1 + \frac{1}{p^2} } < 0 $ OU $x = 1 + 2\sqrt{ 1 + \frac{1}{p^2} } > 0$ \\
sattisfaissent $x \in ]-\infty; 0[ U ]0; + \infty[$

\begin{align*}
    &\text{Si } x < 0, \text{alors il n'y a aucune solution pour (E)} \\
    &\text{Si } p = y' = 0, \text{alors  y: constante alors} y = Cx+K \quad C,K \in \mathbb{R} \\
    &\text{Si } x > 0
\end{align*}
\begin{align*}
    y &= xy' + \sqrt{a+(y')^2} \\
    y &= 1 \pm 2(\sqrt{1+\frac{1}{p^2}})(p) + \sqrt{1+p^2} \\
    y &= 1 \pm 2(\sqrt{p^2+\frac{p^2}{p^2}}) + \sqrt{1+p^2} \\
    y &= 1 \pm 2(\sqrt{p^2+ 1}) + \sqrt{1+p^2}
\end{align*}