L'equation caracteristique (K):
$r^2 -4r +1 = 0$
\begin{align*}
    \Delta &= (-4)^2 - 4(1)(1) \\
    &= 16 -4   \\
    &= 12 > 0  \\
    \sqrt\Delta &= 2\sqrt3
\end{align*}
$r_1 = 2 + \sqrt3$, $r_2 = 2 - \sqrt3$ alors la solution $y_h$ de l'equation homogene est: \\
$Ae^{(2+\sqrt3)x} + Be^{(2-\sqrt3)x}$ / A, B $ \in \mathbb{R}$ 

$f(x) = xe^{2x}$ de la forme ${P_n(x) e^{\delta x}}$ \\
\begin{align*}
    \delta &= 2 \\
    n &= 1 \\
\end{align*}

Donc
\begin{align*}
    y_p &= H_n(x)e^{\delta x} \\
    y_p &= (ax + b)e^{2x} \\, \quad a, b \in \mathbb{R}
\end{align*}

Or 
$y = y_h + y_p$ \\
\boxed{
    y = Ae^{(2+\sqrt3)x} + Be^{(2-\sqrt3)x} + (ax + b)e^{2x}
}